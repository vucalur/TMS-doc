\documentclass[a4paper,12pt,notitlepage]{mwrep}
%%\documentclass[polish,11pt,a4paper,twosides]{article}
%%%\usepackage{fullpage}

\usepackage{./mystyle}

\begin{document}

% footer only for title page:
\newcommand{\cfoottext}{Kraków, \today}
\renewcommand{\footrulewidth}{0.4pt}
\cfoot{\cfoottext}

\begin{titlepage}
\begin{center}
  \resizebox{\textwidth}{!}{\mbox{AKADEMIA GÓRNICZO-HUTNICZA}}\\
  \vspace{2ex}
  \resizebox{\textwidth}{!}{\mbox{Wydział Informatyki, Elektroniki i Telekomunikacji}}\\
  \vspace{4ex}
  \includegraphics[scale=0.15]{images/agh_crop.pdf} \\
  \vspace{4ex}
  \begin{large}KATEDRA INFORMATYKI\end{large} \\
  \vspace{8ex}
  \textbf{\begin{Huge}System prowadzenia zadań w projektach IT\end{Huge}} \\
  \vspace{3ex}
  \textit{\begin{Large}codename: \codename\end{Large}} \\

  \vfill
  \begin{Large}\wersja\end{Large}\\
\end{center}
\begin{large}
  \begin{tabularx}{\textwidth}{lXc}
    \textit{Kierunek, rok studiów} &  & \textit{ } \\
    \hspace{3em}Informatyka, rok III &  &  \\
    \textit{Przedmiot} &  & \\
	\multicolumn{3}{l}{\hspace{3em}Inżynieria Oprogramowania} \\
    \textit{Prowadzący przedmiot} &  & \textit{rok akademicki:} \hfill 2012/2013\\
    \hspace{3em}\minibox{dr inż. Małgorzata Żabińska-Rakoczy\\dr inż. Jarosław Koźlak} &  & \textit{semestr:} \hfill 6\\
  \end{tabularx}
\end{large}


\vspace{2ex}

\noindent
\begin{Large}
Skład zespołu:\end{Large}
\begin{large}
  \newlength{\tblwidth}
  \setlength{\tblwidth}{\textwidth}
  \addtolength{\tblwidth}{-3em}
  \begin{flushright}
    \begin{tabularx}{\tblwidth}{lXr}
      Grzegorz Wilaszek &  &  \\
      Wojciech Krzystek &  &  \\
    \end{tabularx}\end{flushright}
  \end{large}
  \thispagestyle{fancy}
\end{titlepage}



\onehalfspacing

\lhead{\footnotesize{Grzegorz Wilaszek, Wojciech Krzystek}}
\lfoot{\footnotesize{\codename}}
\cfoot{\normalsize Strona \thepage \ z \pageref{LastPage}}
\rfoot{\scriptsize{\wersja}}

\setcounter{secnumdepth}{2}
\setcounter{tocdepth}{2}


\vfill
\begin{center}
\singlespacing
\fbox{\begin{minipage}{0.8\textwidth}
\footnotesize Niniejsze opracowanie powstało w trakcie i jako rezultat zajęć dydaktycznych z przedmiotu 
wymienionego na stronie tytułowej, prowadzonych w Akademii Górniczo-Hutniczej w Krakowie
(AGH) przez osobę (osoby) wymienioną (wymienione) po słowach "Prowadzący przedmiot"
i nie może być wykorzystywane w jakikolwiek sposób i do jakichkolwiek celów, 
w całości lub części, w szczególności publikowane w jakikolwiek sposób
i w jakiejkolwiek formie, bez uzyskania uprzedniej, pisemnej zgody tej
osoby (tych osób) lub odpowiednich władz AGH.
\vspace{2ex} \\
\textbf{Copyright \textcopyright \the\year\ Akademia Górniczo-Hutnicza (AGH) w Krakowie}
      \end{minipage}
}
\onehalfspacing
\end{center}

\tableofcontents

\setcounter{page}{0}
\chapter{Tematyka projektu}
\section{Opis zadania}


\appendix
\chapter*{Zawartość płyty CD}
\begin{description}
	\item[plik dokuentacja.pdf]	 --- plik z tą dokumentacją (dodatkowo dokumentacja jest oddawana w formie drukowanej)
	\item[folder knabees]	 --- folder z kompletnym kodem źródłowym aplikacji
	\item[plik knabees-1.0-jar-with-dependencies.jar]	 --- plik jar zbudowanej aplikacji Javowej
	\item[pokaz.avi]	 --- krótki filmik demonstrujący działanie aplikacji
	\item[folder tests]		--- folder z danymi wejściowymi, oraz wykresami dla przeprowadzonych testów
\end{description}

\addtocounter{page}{-1}

\begin{thebibliography}{9}
\bibitem{enwiki}
	\href{http://en.wikipedia.org/wiki/List_of_knapsack_problems}{http://en.wikipedia.org/wiki/List\_of\_knapsack\_problems}
\bibitem{ba}
	\href{http://www.iaeng.org/publication/IMECS2008/IMECS2008_pp84-88.pdf}{Bee Colony Algorithm for the Multidimensional Knapsack Problem}
\bibitem{swing}
	\href{http://docs.oracle.com/javase/tutorial/uiswing/components/index.html}{Java tutorials, Lesson: Using Swing Components}
\bibitem{mvnex}
	\href{http://www.sonatype.com/Support/Books/Maven-By-Example}{Sonatype, Maven By Example}
\bibitem{mvntcr}
	\href{http://www.sonatype.com/Support/Books/Maven-The-Complete-Reference}{Sonatype, Maven: The Complete Reference}
\bibitem{progit}
	\href{http://git-scm.com/book/}{Scott Chacon, Pro Git}
\bibitem{java2s}
	\href{http://www.java2s.com/Code/Java/Chart/CatalogChart.htm}{Java2S - Programming tutorials and source code examples: Charting}
\end{thebibliography}


\label{LastPage}\phantom{\phantomsection{LastPage}}
\end{document}
